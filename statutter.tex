\documentclass{article}
\usepackage[utf8]{inputenc}

% ::::::::::::::::::::::::::::::::::::::::::::::::::::::::::::::::::::::::::::::
%                           Packages
% ::::::::::::::::::::::::::::::::::::::::::::::::::::::::::::::::::::::::::::::
\usepackage[norsk]{babel}
\usepackage{titlesec} % Edits titles and sections
\usepackage[a4paper]{geometry} % Defines the height and width of the page. 
\usepackage{enumitem} % Package for enumerate lists
\usepackage{hyperref} % Enables clickable hyperlinks
\usepackage{circuitikz} % Enables drawing electrical circuits
\usepackage{graphicx} % Load images
\usepackage{caption} % Enables captions in float environments such as minipages
\usepackage{amsmath} % Improves math environment

% ::::::::::::::::::::::::::::::::::::::::::::::::::::::::::::::::::::::::::::::
%                       Document setup
% ::::::::::::::::::::::::::::::::::::::::::::::::::::::::::::::::::::::::::::::

% Defines the coloring of the urls.
\hypersetup{
	colorlinks=true,
	linkcolor=blue,
	filecolor=magenta,      
	urlcolor=cyan
}

% Redefines the title of the table of contents.
\renewcommand{\contentsname}{Innholdsfortegnelse}

% Declares that there should be no indents when starting new paragraphs.
\setlength{\parindent}{0pt}

% Redefines the formating of sections, subsections and subsubsections.
\titleformat{\section}{\normalfont\huge\bfseries}{\thesection.}{23pt}{}
\titleformat{\subsection}{\normalfont\large\bfseries}{\S\hspace{4pt}\thesubsection}{16pt}{}{}
\titleformat{\subsubsection}[runin]{\normalfont}{\textit{\S\hspace{5pt}\thesubsubsection}}{5pt}{}

\newenvironment{definition}[1][]
    {
        \addtocontents{toc}{\setcounter{tocdepth}{1}}
        \titleformat{\subsection}[runin]{\normalfont}{\textit{\hspace{1pt}\S\hspace{4pt}\thesubsection}}{0pt}{\rule{12pt}{0pt}}{}
        \subsection{}#1
        \begin{minipage}[t]{0.9\linewidth}
    }
    {
        \end{minipage}
     
        \ignorespacesafterend
        \addtocontents{toc}{\setcounter{tocdepth}{2}}
    }

\newenvironment{statute}[1][]
    {
        \titleformat{\subsubsection}[runin]{\normalfont}{\hspace{1pt}\textit{\S\hspace{5pt}\thesubsubsection}}{0pt}{\rule{4pt}{0pt}}{}
        \subsubsection{}#1
        \begin{minipage}[t]{0.9\linewidth}
    }
    {
        \end{minipage}
        
        \ignorespacesafterend
    }

\begin{document}%

    % Frontpage
    \thispagestyle{empty}
        \hfill
        \begin{minipage}{\linewidth}
            \vfill
            \begin{center}
                \rule{\linewidth}{0.1em}\\
                \vspace{0.6\baselineskip}
                {\Huge Statutter}
                \rule{\linewidth}{0.1em}
            \end{center}
            \begin{minipage}{\linewidth}
                \includegraphics[width = \linewidth]{images/MotstandenLogo.jpg}
            \end{minipage}
        \end{minipage}
    \newpage
    
    \tableofcontents
    \newpage

    \section{Definisjoner}
            \begin{definition}
                Studentorchesteret Den Ohmske Motstanden, heretter kalt Motstanden, er orchesterets navn.
            \end{definition}
            \begin{definition}
                Med medlemskap i Linjeforeningen Elektra menes det fullverdige medlemmer og elektroder i linjeforeningen.
            \end{definition}
        	\begin{definition}
        		Med medlemskap i Motstanden menes det: 
        		\begin{itemize}
        			\item Personer som definerer seg selv som medlem av Motstanden, og som innfrir krav til medlemskap i
        			\S\hspace{3pt}\ref{medlemskap}.
        			\item Personer som blir definert av Motstandenstyret til å være medlem av Motstanden, og som innfrir krav til medlemskap i \S\hspace{3pt}\ref{medlemskap}.
        		\end{itemize}
        	\end{definition}
            \begin{definition}
                Lambo er en av orchesterets interne skåler som står skrevet i orchesterets sanghefte.
            \end{definition}
            \begin{definition}[\label{y_faktor_paragraf}]
                Y-variabelen er definert som halvparten av gjennomsnittlig antall oppmøtte medlemmer på de tre foregående øvelsene. Y-variabelen rundes alltid av ned til nærmeste heltall. Den matematiske definisjonene av y-variabelen er uttrykt i ligning \ref{y_faktor_ligning}.
                \begin{equation}\label{y_faktor_ligning}
                    y(n) = \left\lfloor \frac{n}{2*3} \right\rfloor
                \end{equation}
                Hvor:
                \begin{itemize}
                    \item $y(n)$ er y-variabelen.
                    \item $n$ er \textbf{totalt} kumulativt antall oppmøtende medlemmer de tre foregående øvelsene. 
                \end{itemize}
            \end{definition}
            \begin{definition}[\label{overkvalifisert_flertall}]
                Overkvalifisert flertall defineres som mer enn $\frac{2}{\pi}$ oppslutning av antall oppmøtte på generalvorssamlingen.
            \end{definition}

    \section{Orchesteret}
        \subsection{Tilhørighet}
            \begin{statute}
                Motstanden er en interessegruppe underlagt Linjeforeningen Elektra.
            \end{statute}
        
        \subsection{Formål}
            \begin{statute}
                Formålet med Motstanden er å gi medlemmer av Linjeforeningen Elektra et lavterskel musikalsk- sosialtilbud.
        \end{statute}
        
        
        \subsection{Medlemskap}\label{medlemskap}
            \begin{statute}
                For å bli medlem i Motstanden kreves medlemskap i Linjeforeningen Elektra.
            \end{statute}
            \begin{statute}
                Det kreves ingen kompetanse eller forkunnskaper for å være medlem av Motstanden.
            \end{statute}
            \begin{statute}
                Medlemskap i Motstanden er livsvarig, og det er uavhengig av \textbf{utløpt} medlemskap i Linjeforeningen Elektra.            
            \end{statute}
        	\begin{statute}
        		Tre måneder etter første deltagelse på arrangement eller øvelse, kreves det medlemskap i 
        		Motstanden for å videre delta på Motstandens arrangementer og øvelser. Ved spesielle omstendigheter kan dette punktet fravikes dersom det godkjennes av både Motstandenstyret og Elektrastyret. 
        	\end{statute}
        
        \subsection{Utviselse av medlemmer}
        	\begin{statute}[\label{utvisning_av_elektrastyret}]
        		Dersom et medlem blir utvist av linjeforeningen Elektra, skal vedkommende også utvises av Motstanden.
        	\end{statute}
            \begin{statute}[\label{utvisning_av_styret}]
                Et medlem kan utvises ved enstemmig avgjørelse i Motstandens interne styre. Før utvisning skal vedkommende helst få skriftlig advarsel signert av styret.
            \end{statute}
            \begin{statute}
                Vold, seksuell trakassering, og diskriminering er ikke akseptabelt i Motstanden og kan medføre utvisning i henhold til \S\hspace{3pt}\ref{utvisning_av_elektrastyret} eller  \S\hspace{3pt}\ref{utvisning_av_styret}.
            \end{statute}
			\begin{statute}
				Personer som blir utvist fra Motstanden i henhold til \S\hspace{3pt}\ref{utvisning_av_styret} kan kalle inn til medlemsmøte for å stemme over omgjøring av utvisningen. Det kan ikke kalles inn til et slikt møte dersom fornærmede har innsigelse mot å avholde det.
			\end{statute}

        \subsection{Orchesterets protokoller}
            \begin{statute}
                Ved første mulige arrangement Motstanden arrangerer eller deltar på, må samtlige medlemmer ta lambo dersom Ole (eller et annet medlem) knuser et bord.     
            \end{statute}
        
    \section{Uniform og utstyr}
        \subsection{Utlån}
                \begin{statute}
                    Ved utlåning av instrument og utstyr har man selv ansvar for at det kommer tilbake til Motstandens lager ved endt låneperiode. Låneperiode defineres som det tidsintervallet hvor utstyret er utenfor Motstandens lager og fram til det er tilbake til lageret.
                \end{statute}
                \begin{statute}
                    Ved grov uaktsom behandling av instrument og utstyr, kan styret påkreve at vedkommende spiller på vaskebøtte fram til styret ser noe annet hensiktsmessig. 
                \end{statute}

            
        \subsection{Kappen}
                \begin{statute}
                    Kappen skal være grønn. Det skal være svart kant på den nedre kortsiden, og svart kant på begge langsidene. Både utsiden og innsiden skal ha svart kant, og kanten skal være bredere på innsiden. Utsiden av kappen skal være nøytral. Innsiden kan prydes etter eget ønske. Kappen skal aldri vaskes, men den kan fuktes eller bades med.
                \end{statute}

        \subsection{Ølfatlen}
            \begin{statute}
                Ølfatlen kan utformes etter eget ønske, men det skal være en ølfatle. Ved utforming skal ølfatlen være hvit. Ølfatlen skal aldri vaskes, men den kan kokes suppe på.
            \end{statute}
            
        \subsection{Masken}
            \begin{statute}
                Masken skal være grønn og ha svarte kanter. Det er bare tillatt å klippe øyehull når det er ordinær eller ekstraordinær Forohming.
            \end{statute}
           
        \subsection{Hodeplagg}
            \begin{statute}
                Hodeplagg er et obligatorisk tillegg til uniformen. Hodeplaggets utforming er valgfri.
            \end{statute}
            
        \subsection{Forbruksmateriell}
            \begin{statute}
                Forbruksmateriell, som for eksempel saksofonflis og trommestikker, stiller det enkelte medlem med selv.
            \end{statute}
            
        \subsection{Noter}
            \begin{statute}
                Det enkelte medlem har ansvar for å ha egne noter. Notene skal ligge tilgjengelig i Motstandens gjeldende notearkiv.
            \end{statute}
            
    \section{Organisering}
        \subsection{Motstandens titler}
            \begin{statute}[\label{motstandens_titler}]
                Motstanden har følgende titler:
                \begin{enumerate}[font = \bfseries]
                    \item \textbf{Høyimpedant:} Ærestittel for de som har sittet i styret til Motstanden i to eller flere år, og for de som har utøvet ekstraordinær innsats i henhold til \S\hspace{3pt}\ref{stemt_til_høyimpedant}.
                    \item \textbf{Gigaohm:} Sittende- og tidligere ledere for Motstanden.
                    \item \textbf{Megaohm:} Sittende- og tidligere styremedlemmer i Motstanden.
                    \item \textbf{Kiloohm:} De som har vært ohmsk i et år.
                    \item \textbf{Ohm:} Tittel etter gjennomført Forohming.
                    \item \textbf{Kortsluttning:} Tittel før Forohming
                \end{enumerate}
            \end{statute}
            \begin{statute}[\label{stemt_til_høyimpedant}]
            	Ved ekstraordinær innsats kan et medlem bli stemt inn til å bli Høyimpedant på Motstandens generalvorssamling. Avstemningen krever overkvalifisert flertall i henhold til \S\hspace{3pt}\ref{overkvalifisert_flertall}.
            \end{statute}
        	\begin{statute}[\label{statutt_uforohmet_tittel}]
        		Et uforohmet medlem i Motstandend vil alltid være parallellkoblet med en kortslutning. Dette vil si at Motstandens titler (utover kortsluttning) i \S\hspace{3pt}\ref{motstandens_titler} ikke er gjeldende før medlemmet er forohmet. Se figur \ref{figur_uforohment_tittel}.
        	\end{statute}
        
        	\begin{minipage}[H]{0.1\linewidth}
        		\hfill
        	\end{minipage}
      		\begin{minipage}[H]{0.9\linewidth}
      			\captionsetup{type=figure}
      			\centering
	       		\begin{minipage}{0.45\linewidth}
	       			\vspace{\baselineskip}
					\centering
		            \begin{circuitikz}[scale=1.5]
		            	\coordinate (r-medlem-start) at (3,0);
		            	\coordinate (r-medlem-slutt) at (3,2);
		            	\draw (r-medlem-start) 
			            	to [short, o-] (0,0) 
			            	to [R, l_=$R_{\text{tittel}}$] (0,2) 
			            	to [short, -o] (r-medlem-slutt);
		            	\draw (1.5,0) 
		            		to [short] (1.5,2);
		            	\node at (3,1) {$R_{\text{medlem}}=0$};	
		            \end{circuitikz}
		            Motstanden av et \\ \textbf{uforohmet} medlem \\
		            \vspace{\baselineskip}
	        	\end{minipage}
        		\hfill
	        	\begin{minipage}{0.45\linewidth}
	        		\vspace{\baselineskip}
	        		\centering
	        		\begin{circuitikz}[scale=1.5]
	        			\coordinate (r-medlem-start) at (3,0);
	        			\coordinate (r-medlem-slutt) at (3,2);	
	        			\draw (r-medlem-start) 
		        			to [short, o-] (0,0) 
		        			to[R, l_=$R_{\text{tittel}}$] (0,2) 
		        			to [short, -o] (r-medlem-slutt);
	        			\node at (3,1) {$R_{\text{medlem}}=R_{\text{tittel}}$};	
	        		\end{circuitikz}
	        		\\ Motstanden av et \\ \textbf{forohmet} medlem \\
	        		\vspace{\baselineskip}
	        	\end{minipage}
        		\centering
        		\caption{Illustrasjon av \S\hspace{3pt}\ref{statutt_uforohmet_tittel}}
        		\label{figur_uforohment_tittel}
	        \end{minipage}
    
        \subsection{Forohmingen}     
             \begin{statute}
                Forohmingen skal avholdes på høstsemesteret før SMASH.
            \end{statute}
             \begin{statute}
                Ved behov, og hvorvidt det er realistisk gjennomførbart, kan ekstraordinær Forohming avholdes på vårsemesteret før SMASH. Ekstraordinær Forohming planlegges av styret.
             \end{statute}
             
    \section{Styret}
        \subsection{Styrets sammensetning}
            \begin{statute}
                Styret i Motstanden består av:
                \begin{enumerate}[font = \bfseries]
                    \item Leder
                    \item Nestleder
                    \item Økonomiingeniør
                    \item Musikkingeniør
                    \item Sosialingeniør
                    \item Webingeniør
                \end{enumerate}
            \end{statute}

        \subsection{Valg av styret}
            \begin{statute}
                Styret velges på ordinær eller ekstraordinær generalvorssamling.
            \end{statute}
            \begin{statute}
                Styret innehar stillingen fram til neste ordinære eller ekstraordinære generalvorssamling.
            \end{statute}
            
        \subsection{Stillingsbeskrivelse}
            \begin{statute}
                Det gamle styret setter stillingsbeskrivelsen til det nye styret før ordinær generalvorssamling.
            \end{statute}
            
    \section{Generalvorssamling}
        \subsection{Arrangering av generalvorssamling}
            \begin{statute}
                Det skal alltid være tillatt å nyte alkohol på generalvorssamling
            \end{statute}
            \begin{statute} 
                For å avholde generalvorssamling må antall oppmøtende medlemmer overstige y-variabelen som er beskrevet i paragraf \ref{y_faktor_paragraf}.
            \end{statute}
            \begin{statute}
                Generalvorssamling skal avholdes så fort som mulig etter høst-SMASH, senest før november.
            \end{statute}
            
        \subsection{Drikkeprotokoller}
            \begin{statute}
                Det skal drikkes hver gang en statutt vedtas eller avvises på generalvorssamlingen.
            \end{statute}
            \begin{statute}[\label{drikkeprotokoll_forbudte_ord}]
                Det skal drikkes hver gang noen av følgende ord nevnes på generalvorssamlingen: Volleyball, useriøst, seriøst, bord, London, Jørgen. Mellom kl 21:00 og 22:00 skal det drikkes hver gang det nevnes: ja, nei.
            \end{statute}
        	\begin{statute}
        		Statutt \ref{drikkeprotokoll_forbudte_ord} kan ikke omfatte mer en 8 ord.
        	\end{statute}
            
        \subsection{Skåleprotokoller}
            \begin{statute}
                Det skal skåles for alle som blir valgt inn til styret.
            \end{statute}
            \begin{statute}
                Det nye styret tiltrer ikke inn i styrerollen før de har tatt lambo.
            \end{statute}            
        \subsection{Endring av statutter}
            \begin{statute}
                Statuttendringer kan vedtas ved overkvalifisert flertall i henhold til \S\hspace{3pt}  \ref{overkvalifisert_flertall}
            \end{statute}
            
        \subsection{Ekstraordinær generalvorssamling}
            \begin{statute}
                Ekstraordinær generalvorssamling kan avholdes dersom styret ønsker det.
            \end{statute}
            \begin{statute}
                Ekstraordinær generalvorssamling kan avholdes dersom det er ønsket av en medlemsmasse som overstiger y-variabelen definert i paragraf \ref{y_faktor_paragraf}.
            \end{statute}
\end{document}
