  
\section{Generalvorssamling}
    \subsection{Arrangering av generalvorssamling}
        \begin{statute}
            Det skal alltid være tillatt å nyte alkohol på generalvorssamling
        \end{statute}
        \begin{statute} 
            For å avholde generalvorssamling må antall oppmøtende medlemmer overstige y-variabelen som er beskrevet i paragraf \ref{y_faktor_paragraf}.
        \end{statute}
        \begin{statute}
            Generalvorssamling skal avholdes så fort som mulig etter høst-SMASH, senest før november.
        \end{statute}
    
    \subsection{Agenda for generalvorssamlingen}
        \begin{statute}
            Styret i Motstanden plikter å presentere Motstandens regnskap. 
        \end{statute}

    \subsection{Drikkeprotokoller}
        \begin{statute}
            Det skal drikkes hver gang en statutt vedtas eller avvises på generalvorssamlingen.
        \end{statute}
        \begin{statute}
            Ordstyrer har rett til å delegere ut slurker etter eget skjønn.
        \end{statute}
        \begin{statute}[\label{drikkeprotokoll_forbudte_ord}]
            Det skal drikkes hver gang noen av følgende ord nevnes på generalvorssamlingen: Volleyball, useriøst, seriøst, bord, London, saksofon. Mellom kl 21:00 og 22:00 skal det drikkes hver gang det nevnes: ja, nei.
        \end{statute}
        \begin{statute}
            Ordstyrer bestemmer om det skal drikkes eller ikke dersom det oppstår tvilstilfeller vedrørende statutt \ref{drikkeprotokoll_forbudte_ord}.
        \end{statute}
        \begin{statute}
            Statutt \ref{drikkeprotokoll_forbudte_ord} kan ikke omfatte mer enn $\pi^2$ ord.
        \end{statute}
        
    \subsection{Skåleprotokoller}
        \begin{statute}
            Det skal skåles for alle som blir valgt inn til styret.
        \end{statute}
        \begin{statute}
            Det nye styret tiltrer ikke inn i styrerollen før de har tatt lambo.
        \end{statute}            
    \subsection{Endring av statutter}
        \begin{statute}
            Statuttendringer kan vedtas ved overkvalifisert flertall i henhold til \S\hspace{3pt}  \ref{overkvalifisert_flertall}
        \end{statute}
        
    \subsection{Ekstraordinær generalvorssamling}
        \begin{statute}
            Ekstraordinær generalvorssamling kan avholdes dersom styret ønsker det.
        \end{statute}
        \begin{statute}
            Ekstraordinær generalvorssamling kan avholdes dersom det er ønsket av en medlemsmasse som overstiger y-variabelen definert i paragraf \ref{y_faktor_paragraf}.
        \end{statute}

