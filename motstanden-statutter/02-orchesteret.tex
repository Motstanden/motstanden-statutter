\section{Orchesteret}
\subsection{Tilhørighet}
    \begin{statute}
        Motstanden er en interessegruppe underlagt Linjeforeningen Elektra.
    \end{statute}

\subsection{Formål}
    \begin{statute}
        Formålet med Motstanden er å gi medlemmer av Linjeforeningen Elektra et lavterskel musikalsk- sosialtilbud.
    \end{statute}

\subsection{Medlemskap}\label{medlemskap}
    \begin{statute}
        For å bli medlem i Motstanden kreves medlemskap i Linjeforeningen Elektra.
    \end{statute}
    \begin{statute}
        Det kreves ingen kompetanse eller forkunnskaper for å være medlem av Motstanden.
    \end{statute}
    \begin{statute}
        Medlemskap i Motstanden er livsvarig, og det er uavhengig av \textbf{utløpt} medlemskap i Linjeforeningen Elektra.            
    \end{statute}
    \begin{statute}
        Tre måneder etter første deltagelse på arrangement eller øvelse, kreves det medlemskap i 
        Motstanden for å videre delta på Motstandens arrangementer og øvelser. Ved spesielle omstendigheter kan dette punktet fravikes dersom det godkjennes av både Motstandenstyret og Elektrastyret. 
    \end{statute}

\subsection{Utviselse av medlemmer}
    \begin{statute}[\label{utvisning_av_elektrastyret}]
        Dersom et medlem blir utvist av linjeforeningen Elektra, skal vedkommende også utvises av Motstanden.
    \end{statute}
    \begin{statute}[\label{utvisning_av_styret}]
        Et medlem kan utvises ved enstemmig avgjørelse i Motstandens interne styre. Før utvisning skal vedkommende helst få skriftlig advarsel signert av styret.
    \end{statute}
    \begin{statute}
        Vold, seksuell trakassering, og diskriminering er ikke akseptabelt i Motstanden og kan medføre utvisning i henhold til \S\hspace{3pt}\ref{utvisning_av_elektrastyret} eller  \S\hspace{3pt}\ref{utvisning_av_styret}.
    \end{statute}
    \begin{statute}
        Personer som blir utvist fra Motstanden i henhold til \S\hspace{3pt}\ref{utvisning_av_styret} kan kalle inn til medlemsmøte for å stemme over omgjøring av utvisningen. Medlemsmøtet kan foregå som en diskusjon på en øving valgt av den utviste personen. Det kan likevel ikke kalles inn til et slikt møte dersom fornærmede har innsigelse mot å avholde det.
    \end{statute}

\subsection{Orchesterets protokoller}
    \begin{statute}
        Ved første mulige arrangement Motstanden arrangerer eller deltar på, må samtlige medlemmer ta lambo dersom Ole (eller et annet medlem) knuser et bord.     
    \end{statute}

