   
\section{Organisering}
    \subsection{Motstandens titler}
        \begin{statute}[\label{motstandens_titler}]
            Motstanden har følgende titler:
            \begin{enumerate}[font = \bfseries]
                \item \textbf{Høyimpedant:} Ærestittel for de som har sittet i styret til Motstanden i to eller flere år, og for de som har utøvet ekstraordinær innsats i henhold til \S\hspace{3pt}\ref{stemt_til_høyimpedant}.
                \item \textbf{Gigaohm:} Sittende- og tidligere ledere for Motstanden.
                \item \textbf{Megaohm:} Sittende- og tidligere styremedlemmer i Motstanden.
                \item \textbf{Kiloohm:} De som har vært ohmsk i et år.
                \item \textbf{Ohm:} Tittel etter gjennomført Forohming.
                \item \textbf{Kortsluttning:} Tittel før Forohming
            \end{enumerate}
        \end{statute}
        \begin{statute}[\label{stemt_til_høyimpedant}]
            Ved ekstraordinær innsats kan et medlem bli stemt inn til å bli Høyimpedant på Motstandens generalvorssamling. Avstemningen krever overkvalifisert flertall i henhold til \S\hspace{3pt}\ref{overkvalifisert_flertall}.
        \end{statute}
        \begin{statute}[\label{statutt_uforohmet_tittel}]
            Et uforohmet medlem i Motstandend vil alltid være parallellkoblet med en kortslutning. Dette vil si at Motstandens titler (utover kortsluttning) i \S\hspace{3pt}\ref{motstandens_titler} ikke er gjeldende før medlemmet er forohmet. Se figur \ref{figur_uforohment_tittel}.
        \end{statute}
    
        \hfill
          \begin{minipage}[H]{0.9\linewidth}
              \captionsetup{type=figure}
              \centering
               \begin{minipage}{0.475\linewidth}
                   \vspace{\baselineskip}
                \centering
                \begin{circuitikz}[scale=1.5]
                    \coordinate (r-medlem-start) at (3,0);
                    \coordinate (r-medlem-slutt) at (3,2);
                    \draw (r-medlem-start) 
                        to [short, o-] (0,0) 
                        to [R, l_=$R_{\text{tittel}}$] (0,2) 
                        to [short, -o] (r-medlem-slutt);
                    \draw (1.5,0) 
                        to [short] (1.5,2);
                    \node at (3,1) {$R_{\text{medlem}}=0$};	
                \end{circuitikz}
                Motstanden av et \\ \textbf{uforohmet} medlem \\
                \vspace{\baselineskip}
            \end{minipage}
            \hfill
            \begin{minipage}{0.475\linewidth}
                \vspace{\baselineskip}
                \centering
                \begin{circuitikz}[scale=1.5]
                    \coordinate (r-medlem-start) at (3,0);
                    \coordinate (r-medlem-slutt) at (3,2);	
                    \draw (r-medlem-start) 
                        to [short, o-] (0,0) 
                        to[R, l_=$R_{\text{tittel}}$] (0,2) 
                        to [short, -o] (r-medlem-slutt);
                    \node at (3,1) {$R_{\text{medlem}}=R_{\text{tittel}}$};	
                \end{circuitikz}
                \\ Motstanden av et \\ \textbf{forohmet} medlem \\
                \vspace{\baselineskip}
            \end{minipage}
            \centering
            \caption{Illustrasjon av \S\hspace{3pt}\ref{statutt_uforohmet_tittel}}
            \label{figur_uforohment_tittel}
        \end{minipage}

    \subsection{Forohmingen}     
         \begin{statute}
            Forohmingen skal avholdes på høstsemesteret før SMASH.
        \end{statute}
         \begin{statute}
            Ved behov, og hvorvidt det er realistisk gjennomførbart, kan ekstraordinær Forohming avholdes på vårsemesteret før SMASH. Ekstraordinær Forohming planlegges av styret.
         \end{statute}
 