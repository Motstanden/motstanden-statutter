\section{Uniform og utstyr}
    \subsection{Utlån}
        \begin{statute}
            Ved utlåning av instrument og utstyr har man selv ansvar for at det kommer tilbake til Motstandens lager ved endt låneperiode. Låneperiode defineres som det tidsintervallet hvor utstyret er utenfor Motstandens lager og fram til det er tilbake til lageret.
        \end{statute}
        \begin{statute}
            Ved grov uaktsom behandling av instrument og utstyr, kan styret påkreve at vedkommende spiller på vaskebøtte fram til styret ser noe annet hensiktsmessig. 
        \end{statute}

    \subsection{Kappen}
            \begin{statute}
                Kappen skal være grønn. Det skal være svart kant på den nedre kortsiden, og svart kant på begge langsidene. Både utsiden og innsiden skal ha svart kant, og kanten skal være bredere på innsiden. Utsiden av kappen skal være nøytral. Innsiden kan prydes etter eget ønske. Kappen skal aldri vaskes, men den kan fuktes eller bades med.
            \end{statute}

            \begin{statute}
                Kappen skal navngis, og gitt navn skal føres på innsiden av kappekragen etter Forohmingen.
            \end{statute}

    \subsection{Lånekappe}
        \begin{statute}
            Lånekapper er for medlemmer av Motstanden som ikke har laget sin egen kappe. Denne knappen skal ha sølvgaffakant.
        \end{statute}

    \subsection{Ølfatlen}
        \begin{statute}
            Ølfatlen kan utformes etter eget ønske, men det skal være en ølfatle. Ved utforming skal ølfatlen være hvit. Ølfatlen skal aldri vaskes, men den kan kokes suppe på.
        \end{statute}
        
    \subsection{Masken}
        \begin{statute}
            Masken skal være grønn og ha svarte kanter. Det er bare tillatt å klippe øyehull når det er ordinær eller ekstraordinær Forohming.
        \end{statute}
        
    \subsection{Hodeplagg}
        \begin{statute}
            Hodeplagg er et obligatorisk tillegg til uniformen. Hodeplaggets utforming er valgfri.
        \end{statute}
        
    \subsection{Forbruksmateriell}
        \begin{statute}
            Forbruksmateriell, som for eksempel saksofonflis og trommestikker, stiller det enkelte medlem med selv.
        \end{statute}
        
    \subsection{Noter}
        \begin{statute}
            Det enkelte medlem har ansvar for å ha egne noter. Notene skal ligge tilgjengelig i Motstandens gjeldende notearkiv.
        \end{statute}
     